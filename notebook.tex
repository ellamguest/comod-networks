
% Default to the notebook output style

    


% Inherit from the specified cell style.




    
\documentclass[11pt]{article}

    
    
    \usepackage[T1]{fontenc}
    % Nicer default font (+ math font) than Computer Modern for most use cases
    \usepackage{mathpazo}

    % Basic figure setup, for now with no caption control since it's done
    % automatically by Pandoc (which extracts ![](path) syntax from Markdown).
    \usepackage{graphicx}
    % We will generate all images so they have a width \maxwidth. This means
    % that they will get their normal width if they fit onto the page, but
    % are scaled down if they would overflow the margins.
    \makeatletter
    \def\maxwidth{\ifdim\Gin@nat@width>\linewidth\linewidth
    \else\Gin@nat@width\fi}
    \makeatother
    \let\Oldincludegraphics\includegraphics
    % Set max figure width to be 80% of text width, for now hardcoded.
    \renewcommand{\includegraphics}[1]{\Oldincludegraphics[width=.8\maxwidth]{#1}}
    % Ensure that by default, figures have no caption (until we provide a
    % proper Figure object with a Caption API and a way to capture that
    % in the conversion process - todo).
    \usepackage{caption}
    \DeclareCaptionLabelFormat{nolabel}{}
    \captionsetup{labelformat=nolabel}

    \usepackage{adjustbox} % Used to constrain images to a maximum size 
    \usepackage{xcolor} % Allow colors to be defined
    \usepackage{enumerate} % Needed for markdown enumerations to work
    \usepackage{geometry} % Used to adjust the document margins
    \usepackage{amsmath} % Equations
    \usepackage{amssymb} % Equations
    \usepackage{textcomp} % defines textquotesingle
    % Hack from http://tex.stackexchange.com/a/47451/13684:
    \AtBeginDocument{%
        \def\PYZsq{\textquotesingle}% Upright quotes in Pygmentized code
    }
    \usepackage{upquote} % Upright quotes for verbatim code
    \usepackage{eurosym} % defines \euro
    \usepackage[mathletters]{ucs} % Extended unicode (utf-8) support
    \usepackage[utf8x]{inputenc} % Allow utf-8 characters in the tex document
    \usepackage{fancyvrb} % verbatim replacement that allows latex
    \usepackage{grffile} % extends the file name processing of package graphics 
                         % to support a larger range 
    % The hyperref package gives us a pdf with properly built
    % internal navigation ('pdf bookmarks' for the table of contents,
    % internal cross-reference links, web links for URLs, etc.)
    \usepackage{hyperref}
    \usepackage{longtable} % longtable support required by pandoc >1.10
    \usepackage{booktabs}  % table support for pandoc > 1.12.2
    \usepackage[inline]{enumitem} % IRkernel/repr support (it uses the enumerate* environment)
    \usepackage[normalem]{ulem} % ulem is needed to support strikethroughs (\sout)
                                % normalem makes italics be italics, not underlines
    

    
    
    % Colors for the hyperref package
    \definecolor{urlcolor}{rgb}{0,.145,.698}
    \definecolor{linkcolor}{rgb}{.71,0.21,0.01}
    \definecolor{citecolor}{rgb}{.12,.54,.11}

    % ANSI colors
    \definecolor{ansi-black}{HTML}{3E424D}
    \definecolor{ansi-black-intense}{HTML}{282C36}
    \definecolor{ansi-red}{HTML}{E75C58}
    \definecolor{ansi-red-intense}{HTML}{B22B31}
    \definecolor{ansi-green}{HTML}{00A250}
    \definecolor{ansi-green-intense}{HTML}{007427}
    \definecolor{ansi-yellow}{HTML}{DDB62B}
    \definecolor{ansi-yellow-intense}{HTML}{B27D12}
    \definecolor{ansi-blue}{HTML}{208FFB}
    \definecolor{ansi-blue-intense}{HTML}{0065CA}
    \definecolor{ansi-magenta}{HTML}{D160C4}
    \definecolor{ansi-magenta-intense}{HTML}{A03196}
    \definecolor{ansi-cyan}{HTML}{60C6C8}
    \definecolor{ansi-cyan-intense}{HTML}{258F8F}
    \definecolor{ansi-white}{HTML}{C5C1B4}
    \definecolor{ansi-white-intense}{HTML}{A1A6B2}

    % commands and environments needed by pandoc snippets
    % extracted from the output of `pandoc -s`
    \providecommand{\tightlist}{%
      \setlength{\itemsep}{0pt}\setlength{\parskip}{0pt}}
    \DefineVerbatimEnvironment{Highlighting}{Verbatim}{commandchars=\\\{\}}
    % Add ',fontsize=\small' for more characters per line
    \newenvironment{Shaded}{}{}
    \newcommand{\KeywordTok}[1]{\textcolor[rgb]{0.00,0.44,0.13}{\textbf{{#1}}}}
    \newcommand{\DataTypeTok}[1]{\textcolor[rgb]{0.56,0.13,0.00}{{#1}}}
    \newcommand{\DecValTok}[1]{\textcolor[rgb]{0.25,0.63,0.44}{{#1}}}
    \newcommand{\BaseNTok}[1]{\textcolor[rgb]{0.25,0.63,0.44}{{#1}}}
    \newcommand{\FloatTok}[1]{\textcolor[rgb]{0.25,0.63,0.44}{{#1}}}
    \newcommand{\CharTok}[1]{\textcolor[rgb]{0.25,0.44,0.63}{{#1}}}
    \newcommand{\StringTok}[1]{\textcolor[rgb]{0.25,0.44,0.63}{{#1}}}
    \newcommand{\CommentTok}[1]{\textcolor[rgb]{0.38,0.63,0.69}{\textit{{#1}}}}
    \newcommand{\OtherTok}[1]{\textcolor[rgb]{0.00,0.44,0.13}{{#1}}}
    \newcommand{\AlertTok}[1]{\textcolor[rgb]{1.00,0.00,0.00}{\textbf{{#1}}}}
    \newcommand{\FunctionTok}[1]{\textcolor[rgb]{0.02,0.16,0.49}{{#1}}}
    \newcommand{\RegionMarkerTok}[1]{{#1}}
    \newcommand{\ErrorTok}[1]{\textcolor[rgb]{1.00,0.00,0.00}{\textbf{{#1}}}}
    \newcommand{\NormalTok}[1]{{#1}}
    
    % Additional commands for more recent versions of Pandoc
    \newcommand{\ConstantTok}[1]{\textcolor[rgb]{0.53,0.00,0.00}{{#1}}}
    \newcommand{\SpecialCharTok}[1]{\textcolor[rgb]{0.25,0.44,0.63}{{#1}}}
    \newcommand{\VerbatimStringTok}[1]{\textcolor[rgb]{0.25,0.44,0.63}{{#1}}}
    \newcommand{\SpecialStringTok}[1]{\textcolor[rgb]{0.73,0.40,0.53}{{#1}}}
    \newcommand{\ImportTok}[1]{{#1}}
    \newcommand{\DocumentationTok}[1]{\textcolor[rgb]{0.73,0.13,0.13}{\textit{{#1}}}}
    \newcommand{\AnnotationTok}[1]{\textcolor[rgb]{0.38,0.63,0.69}{\textbf{\textit{{#1}}}}}
    \newcommand{\CommentVarTok}[1]{\textcolor[rgb]{0.38,0.63,0.69}{\textbf{\textit{{#1}}}}}
    \newcommand{\VariableTok}[1]{\textcolor[rgb]{0.10,0.09,0.49}{{#1}}}
    \newcommand{\ControlFlowTok}[1]{\textcolor[rgb]{0.00,0.44,0.13}{\textbf{{#1}}}}
    \newcommand{\OperatorTok}[1]{\textcolor[rgb]{0.40,0.40,0.40}{{#1}}}
    \newcommand{\BuiltInTok}[1]{{#1}}
    \newcommand{\ExtensionTok}[1]{{#1}}
    \newcommand{\PreprocessorTok}[1]{\textcolor[rgb]{0.74,0.48,0.00}{{#1}}}
    \newcommand{\AttributeTok}[1]{\textcolor[rgb]{0.49,0.56,0.16}{{#1}}}
    \newcommand{\InformationTok}[1]{\textcolor[rgb]{0.38,0.63,0.69}{\textbf{\textit{{#1}}}}}
    \newcommand{\WarningTok}[1]{\textcolor[rgb]{0.38,0.63,0.69}{\textbf{\textit{{#1}}}}}
    
    
    % Define a nice break command that doesn't care if a line doesn't already
    % exist.
    \def\br{\hspace*{\fill} \\* }
    % Math Jax compatability definitions
    \def\gt{>}
    \def\lt{<}
    % Document parameters
    \title{comod-net-analysis}
    
    
    

    % Pygments definitions
    
\makeatletter
\def\PY@reset{\let\PY@it=\relax \let\PY@bf=\relax%
    \let\PY@ul=\relax \let\PY@tc=\relax%
    \let\PY@bc=\relax \let\PY@ff=\relax}
\def\PY@tok#1{\csname PY@tok@#1\endcsname}
\def\PY@toks#1+{\ifx\relax#1\empty\else%
    \PY@tok{#1}\expandafter\PY@toks\fi}
\def\PY@do#1{\PY@bc{\PY@tc{\PY@ul{%
    \PY@it{\PY@bf{\PY@ff{#1}}}}}}}
\def\PY#1#2{\PY@reset\PY@toks#1+\relax+\PY@do{#2}}

\expandafter\def\csname PY@tok@w\endcsname{\def\PY@tc##1{\textcolor[rgb]{0.73,0.73,0.73}{##1}}}
\expandafter\def\csname PY@tok@c\endcsname{\let\PY@it=\textit\def\PY@tc##1{\textcolor[rgb]{0.25,0.50,0.50}{##1}}}
\expandafter\def\csname PY@tok@cp\endcsname{\def\PY@tc##1{\textcolor[rgb]{0.74,0.48,0.00}{##1}}}
\expandafter\def\csname PY@tok@k\endcsname{\let\PY@bf=\textbf\def\PY@tc##1{\textcolor[rgb]{0.00,0.50,0.00}{##1}}}
\expandafter\def\csname PY@tok@kp\endcsname{\def\PY@tc##1{\textcolor[rgb]{0.00,0.50,0.00}{##1}}}
\expandafter\def\csname PY@tok@kt\endcsname{\def\PY@tc##1{\textcolor[rgb]{0.69,0.00,0.25}{##1}}}
\expandafter\def\csname PY@tok@o\endcsname{\def\PY@tc##1{\textcolor[rgb]{0.40,0.40,0.40}{##1}}}
\expandafter\def\csname PY@tok@ow\endcsname{\let\PY@bf=\textbf\def\PY@tc##1{\textcolor[rgb]{0.67,0.13,1.00}{##1}}}
\expandafter\def\csname PY@tok@nb\endcsname{\def\PY@tc##1{\textcolor[rgb]{0.00,0.50,0.00}{##1}}}
\expandafter\def\csname PY@tok@nf\endcsname{\def\PY@tc##1{\textcolor[rgb]{0.00,0.00,1.00}{##1}}}
\expandafter\def\csname PY@tok@nc\endcsname{\let\PY@bf=\textbf\def\PY@tc##1{\textcolor[rgb]{0.00,0.00,1.00}{##1}}}
\expandafter\def\csname PY@tok@nn\endcsname{\let\PY@bf=\textbf\def\PY@tc##1{\textcolor[rgb]{0.00,0.00,1.00}{##1}}}
\expandafter\def\csname PY@tok@ne\endcsname{\let\PY@bf=\textbf\def\PY@tc##1{\textcolor[rgb]{0.82,0.25,0.23}{##1}}}
\expandafter\def\csname PY@tok@nv\endcsname{\def\PY@tc##1{\textcolor[rgb]{0.10,0.09,0.49}{##1}}}
\expandafter\def\csname PY@tok@no\endcsname{\def\PY@tc##1{\textcolor[rgb]{0.53,0.00,0.00}{##1}}}
\expandafter\def\csname PY@tok@nl\endcsname{\def\PY@tc##1{\textcolor[rgb]{0.63,0.63,0.00}{##1}}}
\expandafter\def\csname PY@tok@ni\endcsname{\let\PY@bf=\textbf\def\PY@tc##1{\textcolor[rgb]{0.60,0.60,0.60}{##1}}}
\expandafter\def\csname PY@tok@na\endcsname{\def\PY@tc##1{\textcolor[rgb]{0.49,0.56,0.16}{##1}}}
\expandafter\def\csname PY@tok@nt\endcsname{\let\PY@bf=\textbf\def\PY@tc##1{\textcolor[rgb]{0.00,0.50,0.00}{##1}}}
\expandafter\def\csname PY@tok@nd\endcsname{\def\PY@tc##1{\textcolor[rgb]{0.67,0.13,1.00}{##1}}}
\expandafter\def\csname PY@tok@s\endcsname{\def\PY@tc##1{\textcolor[rgb]{0.73,0.13,0.13}{##1}}}
\expandafter\def\csname PY@tok@sd\endcsname{\let\PY@it=\textit\def\PY@tc##1{\textcolor[rgb]{0.73,0.13,0.13}{##1}}}
\expandafter\def\csname PY@tok@si\endcsname{\let\PY@bf=\textbf\def\PY@tc##1{\textcolor[rgb]{0.73,0.40,0.53}{##1}}}
\expandafter\def\csname PY@tok@se\endcsname{\let\PY@bf=\textbf\def\PY@tc##1{\textcolor[rgb]{0.73,0.40,0.13}{##1}}}
\expandafter\def\csname PY@tok@sr\endcsname{\def\PY@tc##1{\textcolor[rgb]{0.73,0.40,0.53}{##1}}}
\expandafter\def\csname PY@tok@ss\endcsname{\def\PY@tc##1{\textcolor[rgb]{0.10,0.09,0.49}{##1}}}
\expandafter\def\csname PY@tok@sx\endcsname{\def\PY@tc##1{\textcolor[rgb]{0.00,0.50,0.00}{##1}}}
\expandafter\def\csname PY@tok@m\endcsname{\def\PY@tc##1{\textcolor[rgb]{0.40,0.40,0.40}{##1}}}
\expandafter\def\csname PY@tok@gh\endcsname{\let\PY@bf=\textbf\def\PY@tc##1{\textcolor[rgb]{0.00,0.00,0.50}{##1}}}
\expandafter\def\csname PY@tok@gu\endcsname{\let\PY@bf=\textbf\def\PY@tc##1{\textcolor[rgb]{0.50,0.00,0.50}{##1}}}
\expandafter\def\csname PY@tok@gd\endcsname{\def\PY@tc##1{\textcolor[rgb]{0.63,0.00,0.00}{##1}}}
\expandafter\def\csname PY@tok@gi\endcsname{\def\PY@tc##1{\textcolor[rgb]{0.00,0.63,0.00}{##1}}}
\expandafter\def\csname PY@tok@gr\endcsname{\def\PY@tc##1{\textcolor[rgb]{1.00,0.00,0.00}{##1}}}
\expandafter\def\csname PY@tok@ge\endcsname{\let\PY@it=\textit}
\expandafter\def\csname PY@tok@gs\endcsname{\let\PY@bf=\textbf}
\expandafter\def\csname PY@tok@gp\endcsname{\let\PY@bf=\textbf\def\PY@tc##1{\textcolor[rgb]{0.00,0.00,0.50}{##1}}}
\expandafter\def\csname PY@tok@go\endcsname{\def\PY@tc##1{\textcolor[rgb]{0.53,0.53,0.53}{##1}}}
\expandafter\def\csname PY@tok@gt\endcsname{\def\PY@tc##1{\textcolor[rgb]{0.00,0.27,0.87}{##1}}}
\expandafter\def\csname PY@tok@err\endcsname{\def\PY@bc##1{\setlength{\fboxsep}{0pt}\fcolorbox[rgb]{1.00,0.00,0.00}{1,1,1}{\strut ##1}}}
\expandafter\def\csname PY@tok@kc\endcsname{\let\PY@bf=\textbf\def\PY@tc##1{\textcolor[rgb]{0.00,0.50,0.00}{##1}}}
\expandafter\def\csname PY@tok@kd\endcsname{\let\PY@bf=\textbf\def\PY@tc##1{\textcolor[rgb]{0.00,0.50,0.00}{##1}}}
\expandafter\def\csname PY@tok@kn\endcsname{\let\PY@bf=\textbf\def\PY@tc##1{\textcolor[rgb]{0.00,0.50,0.00}{##1}}}
\expandafter\def\csname PY@tok@kr\endcsname{\let\PY@bf=\textbf\def\PY@tc##1{\textcolor[rgb]{0.00,0.50,0.00}{##1}}}
\expandafter\def\csname PY@tok@bp\endcsname{\def\PY@tc##1{\textcolor[rgb]{0.00,0.50,0.00}{##1}}}
\expandafter\def\csname PY@tok@fm\endcsname{\def\PY@tc##1{\textcolor[rgb]{0.00,0.00,1.00}{##1}}}
\expandafter\def\csname PY@tok@vc\endcsname{\def\PY@tc##1{\textcolor[rgb]{0.10,0.09,0.49}{##1}}}
\expandafter\def\csname PY@tok@vg\endcsname{\def\PY@tc##1{\textcolor[rgb]{0.10,0.09,0.49}{##1}}}
\expandafter\def\csname PY@tok@vi\endcsname{\def\PY@tc##1{\textcolor[rgb]{0.10,0.09,0.49}{##1}}}
\expandafter\def\csname PY@tok@vm\endcsname{\def\PY@tc##1{\textcolor[rgb]{0.10,0.09,0.49}{##1}}}
\expandafter\def\csname PY@tok@sa\endcsname{\def\PY@tc##1{\textcolor[rgb]{0.73,0.13,0.13}{##1}}}
\expandafter\def\csname PY@tok@sb\endcsname{\def\PY@tc##1{\textcolor[rgb]{0.73,0.13,0.13}{##1}}}
\expandafter\def\csname PY@tok@sc\endcsname{\def\PY@tc##1{\textcolor[rgb]{0.73,0.13,0.13}{##1}}}
\expandafter\def\csname PY@tok@dl\endcsname{\def\PY@tc##1{\textcolor[rgb]{0.73,0.13,0.13}{##1}}}
\expandafter\def\csname PY@tok@s2\endcsname{\def\PY@tc##1{\textcolor[rgb]{0.73,0.13,0.13}{##1}}}
\expandafter\def\csname PY@tok@sh\endcsname{\def\PY@tc##1{\textcolor[rgb]{0.73,0.13,0.13}{##1}}}
\expandafter\def\csname PY@tok@s1\endcsname{\def\PY@tc##1{\textcolor[rgb]{0.73,0.13,0.13}{##1}}}
\expandafter\def\csname PY@tok@mb\endcsname{\def\PY@tc##1{\textcolor[rgb]{0.40,0.40,0.40}{##1}}}
\expandafter\def\csname PY@tok@mf\endcsname{\def\PY@tc##1{\textcolor[rgb]{0.40,0.40,0.40}{##1}}}
\expandafter\def\csname PY@tok@mh\endcsname{\def\PY@tc##1{\textcolor[rgb]{0.40,0.40,0.40}{##1}}}
\expandafter\def\csname PY@tok@mi\endcsname{\def\PY@tc##1{\textcolor[rgb]{0.40,0.40,0.40}{##1}}}
\expandafter\def\csname PY@tok@il\endcsname{\def\PY@tc##1{\textcolor[rgb]{0.40,0.40,0.40}{##1}}}
\expandafter\def\csname PY@tok@mo\endcsname{\def\PY@tc##1{\textcolor[rgb]{0.40,0.40,0.40}{##1}}}
\expandafter\def\csname PY@tok@ch\endcsname{\let\PY@it=\textit\def\PY@tc##1{\textcolor[rgb]{0.25,0.50,0.50}{##1}}}
\expandafter\def\csname PY@tok@cm\endcsname{\let\PY@it=\textit\def\PY@tc##1{\textcolor[rgb]{0.25,0.50,0.50}{##1}}}
\expandafter\def\csname PY@tok@cpf\endcsname{\let\PY@it=\textit\def\PY@tc##1{\textcolor[rgb]{0.25,0.50,0.50}{##1}}}
\expandafter\def\csname PY@tok@c1\endcsname{\let\PY@it=\textit\def\PY@tc##1{\textcolor[rgb]{0.25,0.50,0.50}{##1}}}
\expandafter\def\csname PY@tok@cs\endcsname{\let\PY@it=\textit\def\PY@tc##1{\textcolor[rgb]{0.25,0.50,0.50}{##1}}}

\def\PYZbs{\char`\\}
\def\PYZus{\char`\_}
\def\PYZob{\char`\{}
\def\PYZcb{\char`\}}
\def\PYZca{\char`\^}
\def\PYZam{\char`\&}
\def\PYZlt{\char`\<}
\def\PYZgt{\char`\>}
\def\PYZsh{\char`\#}
\def\PYZpc{\char`\%}
\def\PYZdl{\char`\$}
\def\PYZhy{\char`\-}
\def\PYZsq{\char`\'}
\def\PYZdq{\char`\"}
\def\PYZti{\char`\~}
% for compatibility with earlier versions
\def\PYZat{@}
\def\PYZlb{[}
\def\PYZrb{]}
\makeatother


    % Exact colors from NB
    \definecolor{incolor}{rgb}{0.0, 0.0, 0.5}
    \definecolor{outcolor}{rgb}{0.545, 0.0, 0.0}



    
    % Prevent overflowing lines due to hard-to-break entities
    \sloppy 
    % Setup hyperref package
    \hypersetup{
      breaklinks=true,  % so long urls are correctly broken across lines
      colorlinks=true,
      urlcolor=urlcolor,
      linkcolor=linkcolor,
      citecolor=citecolor,
      }
    % Slightly bigger margins than the latex defaults
    
    \geometry{verbose,tmargin=1in,bmargin=1in,lmargin=1in,rmargin=1in}
    
    

    \begin{document}
    
    
    \maketitle
    
    

    
    \section{Research Context}\label{research-context}

Online communities often have the freedom, and responsibility, to define
their own community norms. The social news site Reddit is an example of
a platform that imposes no editorial control over content produced.
Reddit's administrators instead encourage users to create topic-based
forums, called subreddits, according to their own desires and to develop
unique standards of acceptable behaviours. A small number of community
members who volunteer as moderators together with the wider community of
non-moderating contributors work to define the purpose and tone of their
subreddits, and detail a code of conduct for participating in the space.
These developments are made both implicitly and explicitly and can
highlight irreconcilable desires within the community. As such Reddit
makes a fascinating subject of study for gaining understanding of the
techniques and methods members and moderators of online communities use
in the delicate process of norm-setting.

This research is in the early stages of defining the theoretical and
methodological considerations required in researching the processes of
norm-setting by subreddit communities. I am currently piloting this work
by studying two distinct subreddits; r/The\_Donald - a community for
ardent supporters of Donald Trump - and r/changemyview -- a forum where
users actively encourage others to try to change their opinions on any
given topic. I am seeking to identify changes in the respective
moderator networks of r/The\_Donald and r/ChangeMyView moderators over
time, and how possible `moderation eras' correspond with different eras
of community norms and standards within each community.

These particular communities are of substantive interest because they
highlight the variety of ways in which subreddits can operate, and the
kind of positioning subreddits can have within the wider reddit
community. r/The\_Donald operates as a safe space, or echo chamber, for
supporters of Donald Trump, any dissenting opinions are banned. At the
same time, the subreddit's moderators have a fraught relationship with
reddit employees, the `admins', who they accuse of censoring
r/The\_Donald content from the wider platform. This juxtaposition raises
interesting theoretical considerations for the meaning of free speech on
the platform. Conversely, r/changemyview exists as a space for rational,
well evidenced and argued exchanges of opinion. Posts are rarely banned
for what content they say but for whether it is contributed in
accordance with strict submission standards.

    \begin{Verbatim}[commandchars=\\\{\}]
{\color{incolor}In [{\color{incolor}3}]:} \PY{k+kn}{from} \PY{n+nn}{functions} \PY{k}{import} \PY{o}{*}
        \PY{k+kn}{from} \PY{n+nn}{mod\PYZus{}timeline\PYZus{}figures} \PY{k}{import} \PY{o}{*}
        \PY{o}{\PYZpc{}}\PY{k}{matplotlib} inline
        
        \PY{l+s+sd}{\PYZsq{}\PYZsq{}\PYZsq{}INITIALIZE DATA\PYZsq{}\PYZsq{}\PYZsq{}}
        \PY{n}{date} \PY{o}{=} \PY{l+s+s1}{\PYZsq{}}\PY{l+s+s1}{2017\PYZhy{}10\PYZhy{}27}\PY{l+s+s1}{\PYZsq{}}
        \PY{n}{output} \PY{o}{=} \PY{n}{output\PYZus{}dict}\PY{p}{(}\PY{n}{date}\PY{p}{)}
        
        \PY{l+s+sd}{\PYZsq{}\PYZsq{}\PYZsq{}output[\PYZsq{}td\PYZsq{}][\PYZsq{}desc\PYZus{}table\PYZsq{}] = output[\PYZsq{}td\PYZsq{}][\PYZsq{}desc\PYZus{}table\PYZsq{}].reindex([\PYZsq{}\PYZsh{} nodes\PYZsq{},\PYZsq{}\PYZsh{} edges\PYZsq{}, \PYZsq{}\PYZsh{} components\PYZsq{},}
        \PY{l+s+sd}{                                             \PYZsq{}\PYZsh{} isolates\PYZsq{}, \PYZsq{}density\PYZsq{},\PYZsq{}EI index\PYZsq{},\PYZsq{}\PYZsh{} BM partitions\PYZsq{}])}
        \PY{l+s+sd}{output[\PYZsq{}cmv\PYZsq{}][\PYZsq{}desc\PYZus{}table\PYZsq{}] = output[\PYZsq{}cmv\PYZsq{}][\PYZsq{}desc\PYZus{}table\PYZsq{}].reindex([\PYZsq{}\PYZsh{} nodes\PYZsq{},\PYZsq{}\PYZsh{} edges\PYZsq{}, \PYZsq{}\PYZsh{} components\PYZsq{},}
        \PY{l+s+sd}{                                             \PYZsq{}\PYZsh{} isolates\PYZsq{}, \PYZsq{}density\PYZsq{},\PYZsq{}EI index\PYZsq{},\PYZsq{}\PYZsh{} BM partitions\PYZsq{}])}
        \PY{l+s+sd}{\PYZsq{}\PYZsq{}\PYZsq{}}
        
        \PY{n}{sub1} \PY{o}{=} \PY{l+s+s1}{\PYZsq{}}\PY{l+s+s1}{td}\PY{l+s+s1}{\PYZsq{}}
        \PY{n}{subname1} \PY{o}{=} \PY{l+s+s1}{\PYZsq{}}\PY{l+s+s1}{r/The\PYZus{}Donald}\PY{l+s+s1}{\PYZsq{}}
        \PY{c+c1}{\PYZsh{}mods1 = output[sub1][\PYZsq{}mods\PYZsq{}]}
        \PY{c+c1}{\PYZsh{}subs1 = output[sub1][\PYZsq{}subs\PYZsq{}]}
        
        \PY{n}{sub2} \PY{o}{=} \PY{l+s+s1}{\PYZsq{}}\PY{l+s+s1}{cmv}\PY{l+s+s1}{\PYZsq{}}
        \PY{n}{subname2} \PY{o}{=} \PY{l+s+s1}{\PYZsq{}}\PY{l+s+s1}{r/changemyview}\PY{l+s+s1}{\PYZsq{}}
        \PY{c+c1}{\PYZsh{}mods2 = output[sub2][\PYZsq{}mods\PYZsq{}]}
        \PY{c+c1}{\PYZsh{}subs2 = output[sub2][\PYZsq{}subs\PYZsq{}]}
\end{Verbatim}


    \begin{Verbatim}[commandchars=\\\{\}]
{\color{incolor}In [{\color{incolor}4}]:} \PY{l+s+sd}{\PYZsq{}\PYZsq{}\PYZsq{}UPDATE PLOTS\PYZsq{}\PYZsq{}\PYZsq{}}
        \PY{n}{mod\PYZus{}count\PYZus{}plots}\PY{p}{(}\PY{n}{output}\PY{p}{[}\PY{l+s+s1}{\PYZsq{}}\PY{l+s+s1}{td}\PY{l+s+s1}{\PYZsq{}}\PY{p}{]}\PY{p}{)}
        \PY{n}{mod\PYZus{}count\PYZus{}plots}\PY{p}{(}\PY{n}{output}\PY{p}{[}\PY{l+s+s1}{\PYZsq{}}\PY{l+s+s1}{cmv}\PY{l+s+s1}{\PYZsq{}}\PY{p}{]}\PY{p}{)}
        \PY{n}{mod\PYZus{}count\PYZus{}headers} \PY{o}{=} \PY{p}{[}\PY{l+s+s1}{\PYZsq{}}\PY{l+s+s1}{Mod Type}\PY{l+s+s1}{\PYZsq{}}\PY{p}{,} \PY{l+s+s1}{\PYZsq{}}\PY{l+s+s1}{all\PYZus{}mods}\PY{l+s+s1}{\PYZsq{}}\PY{p}{,}\PY{l+s+s1}{\PYZsq{}}\PY{l+s+s1}{active\PYZus{}mods}\PY{l+s+s1}{\PYZsq{}}\PY{p}{,} \PY{l+s+s1}{\PYZsq{}}\PY{l+s+s1}{diff\PYZus{}}\PY{l+s+s1}{\PYZpc{}}\PY{l+s+s1}{\PYZsq{}}\PY{p}{]}
        
        \PY{n}{td\PYZus{}timeline}\PY{p}{(}\PY{p}{)}
        \PY{n}{cmv\PYZus{}timeline}\PY{p}{(}\PY{p}{)}
        
        \PY{n}{twomode\PYZus{}net\PYZus{}plot}\PY{p}{(}\PY{n}{output}\PY{p}{,} \PY{l+s+s1}{\PYZsq{}}\PY{l+s+s1}{td}\PY{l+s+s1}{\PYZsq{}}\PY{p}{)}
        \PY{n}{mod\PYZus{}net\PYZus{}plot}\PY{p}{(}\PY{n}{output}\PY{p}{,} \PY{l+s+s1}{\PYZsq{}}\PY{l+s+s1}{td}\PY{l+s+s1}{\PYZsq{}}\PY{p}{)}
        \PY{n}{sub\PYZus{}net\PYZus{}plot}\PY{p}{(}\PY{n}{output}\PY{p}{,} \PY{l+s+s1}{\PYZsq{}}\PY{l+s+s1}{td}\PY{l+s+s1}{\PYZsq{}}\PY{p}{)}
        
        \PY{n}{twomode\PYZus{}net\PYZus{}plot}\PY{p}{(}\PY{n}{output}\PY{p}{,} \PY{l+s+s1}{\PYZsq{}}\PY{l+s+s1}{cmv}\PY{l+s+s1}{\PYZsq{}}\PY{p}{)}
        \PY{n}{mod\PYZus{}net\PYZus{}plot}\PY{p}{(}\PY{n}{output}\PY{p}{,} \PY{l+s+s1}{\PYZsq{}}\PY{l+s+s1}{cmv}\PY{l+s+s1}{\PYZsq{}}\PY{p}{)}
        \PY{n}{sub\PYZus{}net\PYZus{}plot}\PY{p}{(}\PY{n}{output}\PY{p}{,} \PY{l+s+s1}{\PYZsq{}}\PY{l+s+s1}{cmv}\PY{l+s+s1}{\PYZsq{}}\PY{p}{)}
\end{Verbatim}


    \section{Introduction - Moderator
Counts}\label{introduction---moderator-counts}

The complete co-moderation network for each subreddit includes all
redditors to have ever been a moderator of the ego subreddit. However,
for the purposes of the network analysis I am only interested in those
redditors who still moderate at least one subreddit. Thus redditors who
do not moderate at al at the time of data collection, or whose account
are no longer active are removed from the data for network analysis. I
refer the the subset of currently moderating redditors as `active mods'.

\subsubsection{Moderator Types}\label{moderator-types}

There exists a strict hierarchy of moderator permissions types and
seniority across all subreddits. For the purposes of the current
research I do not address the full variety of permissions types, but
dichotomise between \textbf{`top'} moderators -- those with full
permissions, and \textbf{`non-tops'}, those without. The most important
benefit of being a top moderator is having the ability to add or remove
other moderators. All top moderators share the same abilities except in
the case of changing the permissions of other moderators. A top
moderator can only change the permissions of, or remove as moderator, a
moderator younger than themselves. A strict time-based seniority exists
in this way regardless of permissions levels.

I have also separated moderators into those who currently moderator the
ego subreddit and those who formerly moderated the ego subreddit. Thus
there are four categories of moderator types:

\begin{enumerate}
\def\labelenumi{\arabic{enumi}.}
\tightlist
\item
  \textbf{Current tops} - those who are currently moderators in the ego
  subreddit, and have at any point had full moderator permissions
\item
  \textbf{Current non-tops} - those are current moderators and have
  never had full permissions
\item
  \textbf{Former tops} - those who do not currently moderate the ego
  subreddit but at one point had full moderator permisisons
\item
  \textbf{Former non-tops} - those who formerly moderated the ego
  subreddit but never had full permiissions
\end{enumerate}

It is of interest to see how many mods, and of which mod type, are no
longer active.

\subsubsection{Moderator Activity
Counts}\label{moderator-activity-counts}

The following tables and plots compare the breakdown of \emph{total} and
\emph{active} moderators by mod type.

\begin{verbatim}
<tr>
    <th  style="text-align: center">{{subname1}} moderator counts</th>
    <th  style="text-align: center">{{subname1}} mod type counts</th>
</tr>
<tr>
    <td>{{output[sub1]['mod_counts'][mod_count_headers].T}}</td>
    <td><img src="td_mod_count_plots.png"></td>

</tr>
\end{verbatim}

    83\% of former non-top moderators of \{\{subname1\}\} no longer moderate
any subreddits, compared to only 17\% of former tops. This could be due
to the existence of 'dummy accounts', which were only created for a
brief purpose and were later deactivated.

\begin{verbatim}
<tr>
    <th  style="text-align: center">{{subname2}} moderator counts</th>
    <th  style="text-align: center">{{subname2}} moderator counts</th>
</tr>
<tr>
    <td>{{output[sub2]['mod_counts'][mod_count_headers]}}</td>
    <td><img src="cmv_mod_count_plots.png"></td>
</tr>
\end{verbatim}

    Comparatively for \{\{subname2\}\}, 41\% of former non-tops and 58\% of
former tops currently do not moderator any subreddits. However, given
the small sample numbers for each mod type in \{\{subname2\}\}

From the figures it is clear that the majority of redditors to ever
moderate \{\{subname1\}\} are former non-tops while for \{\{subname2\}\}
the majority are current tops. This suggests two differences in the
moderator network.

\begin{enumerate}
\def\labelenumi{\arabic{enumi})}
\tightlist
\item
  \textbf{Moderation tenure is more unstable in \{\{subname1\}\}.}
\end{enumerate}

Though \{\{subname2\}\} is older, \{\{subname1\}\} has had \emph{check
number - about 3 times as many??} moderators and the majority of these
moderators have left the network. This can be seem more clearly in the
moderation timelines in the next section.

\begin{enumerate}
\def\labelenumi{\arabic{enumi})}
\setcounter{enumi}{1}
\tightlist
\item
  \textbf{Moderation power is more dispersed in \{\{subname2\}\}.}
\end{enumerate}

The majority of mods to ever appear in the \{\{subname2\}\} network are
or have been top moderators \emph{add percentage}. In \{\{subname1\}\}
only a fraction of moderators have had full permissions *add
percentage**.

*Lead into how the co-mod networks will explore this

\section{Next section - Timeline of moderator presences and
permissions}\label{next-section---timeline-of-moderator-presences-and-permissions}

The first part of my research examines the changes in the group of
moderators in each subreddit over time. I am particularly interested in
when moderators entered/left/re-entered and what level of permissions
they held at anytime. By first understanding the patterns of the
moderation group, my aim is to ultimately address the following general
research questions:

\begin{itemize}
\tightlist
\item
  How much influence do moderators have on the norm-setting process of
  subreddits?
\item
  Is this influence administered in a more implicit or explicit manner?
\item
  Do these relate to certain characteristics of the subreddit, in
  particular content?
\end{itemize}

\paragraph{Hypothesis 1A: Changes in top moderator(s) will be followed
by changes in the overall moderator groups as non-top mods join or leave
in support of
leaders}\label{hypothesis-1a-changes-in-top-moderators-will-be-followed-by-changes-in-the-overall-moderator-groups-as-non-top-mods-join-or-leave-in-support-of-leaders}

To be effective mod groups will need at least one member to be a top
moderator, ideally a senior top mod. This hypothesis tests whether
moderators do work in groups, and whether these groups are led by mods
with greater powers When a group of moderators join or leave together
this will be called a `moderation era'

\paragraph{Hypothesis 1B: These `moderation eras' will correspond with
changes in the a) norm standards and b) content of the
subreddit.}\label{hypothesis-1b-these-moderation-eras-will-correspond-with-changes-in-the-a-norm-standards-and-b-content-of-the-subreddit.}

If moderation groups are found, as being part of the same moderation
era, this hypothesis tests to what extent the groups shares a common
interest for the subreddit If groups share reflected in the other
subreddits they choose to moderate. This leads to the analysis of the
co-moderation network

\section{Following section - comoderation
networks}\label{following-section---comoderation-networks}

The second stage of the research is to examine the co-moderation network
for each subreddit. The co-moderation network is a one-mode projection
for the two-mode moderator by subreddits network. Moderators are tied if
they currently moderate a shared subreddit other than CMV or TD,
respectively. On the assumption that moderators have some level of
influence over their subreddits, I am seeking to determine whether types
of influence are correlated with features of the co-moderation network.
Example general research questions are:

\begin{itemize}
\tightlist
\item
  Do the most influential moderators:

  \begin{itemize}
  \tightlist
  \item
    moderator many disparate communities?
  \item
    show niche moderating interests?
  \item
    or exclusively moderator a single community?
  \end{itemize}
\item
  Following from Hypothesis 1B:

  \begin{itemize}
  \tightlist
  \item
    Do moderators act together in groups?
  \item
    Do these groups reflect content or form based cohesion?
  \item
    Do moderators group to impose their influence together?
  \end{itemize}
\end{itemize}

\paragraph{Hypothesis 2A: Redditors who moderate the target subreddit
during the same period will tend to also co-moderator other
subreddits}\label{hypothesis-2a-redditors-who-moderate-the-target-subreddit-during-the-same-period-will-tend-to-also-co-moderator-other-subreddits}

This hypothesis tests whether groups of moderators determined by
moderation eras show homophily in terms of content interests.

\paragraph{Hypothesis 2B: Those subreddit clusters will be indicative of
mods' content interests for the target
subreddit.}\label{hypothesis-2b-those-subreddit-clusters-will-be-indicative-of-mods-content-interests-for-the-target-subreddit.}

If it is determined that moderation era groups do have similar
interests, I will then begin looking at the changes in implicit and
explicit norms in the subreddits in the third part of this research.

\subsection{r/The\_Donald Part 1 -
Timeline}\label{rthe_donald-part-1---timeline}

\subsubsection{Moderator Attributes and
Eras}\label{moderator-attributes-and-eras}

Moderator lines are coloured by whether they are a current (red) or
former (blue) moderator. Time sections when a moderator had top
moderator permissions are shown in a darker shade of red or blue,
respectively. The timeline shows that the group of moderators has been
unstable as only a fraction of all moderators are still present and many
moderators have left and returned (shown by the number of broken lines).
New moderators are added in chunks at various time points, often
following the making of a new `top' moderator, one with full
permissions. This offers some support for hypothesis 1A. However not all
moderation eras are preceded by top moderators, some appear to be
related to offline and online events.

\subsubsection{Event reference lines}\label{event-reference-lines}

The four event reference lines on the timeline mark two offline (in
black) and two online (in green) events. The offline, real world events
are the date of the 2016 US Presidential election and the 2017
inauguration of Donald Trump as president. A few moderators entered, or
re-entered the timeline at the time of Trump's election and a large
group entered around a week after the inauguration. This suggests that
changes in the group of moderators may correspond with real world Trump
related events, especially as new or returning Trump supporters respond
to his real world successes. The green lines correspond with events that
highlight the strained and controversial relationship between TD
moderators and the employees of reddit, know as the admins. This
suggests that subreddit members may react to the perceived external
threat by admins to become moderators of the subreddit. This may have
interesting implications for the way in which these moderators seek to
run the subreddit.

\{\{subname1\}\} Presence Timeline

Moderator Stability

The timeline shows that no current moderators of \{\{subname1\}\} have
been in the network since the subreddit was created. Some of the current
non-tops joined together around \emph{time?} and only brielfy left the
network at times of fracture \emph{times w/ white lines}

In particular, most of the current top moderators have also had
fractured history lines, as they have either left the network at some
point and/or previously held a non-top position.

Most important to note are the multiple shelf like features in the
timeline. These periods of time when many, if not most, moderators were
removed or added as moderators en masse, in a brief period of time. This
could only be done by a senior top mod. I posit that these mass mod
migrations occur when the leading top mod(s) want to restructure the
moderator in some way, or when the most senior top mod hands over power
to another moderator.

    \subsection{r/changemyview Part 1 -
Timeline}\label{rchangemyview-part-1---timeline}

The timeline for r/changemyview (CMV) shows that the moderator network
is much more stable and less hierarchical compared to TD. New moderators
have been added at a steady rate. There are two time points when groups
of moderators left together, around June 2015 and February 2017. These
will likely correspond with times of explicit change within the
community. Otherwise the most notable finding is the three short
moderation eras, which occurred around July 2013 and December 2013, when
the same group of about 10 moderators were briefly added together three
separate times. As it was in the early days of the subreddit, I expect
this to correspond with changes in the explicit norm-setting process of
the subreddit.

\{\{subname2\}\} Moderator Presence Timeline

    \subsection{Part 2 - Co-Moderation
Networks}\label{part-2---co-moderation-networks}

The following table shows the breakdown of current and former CMV mods
by current moderation status. Only 15\% of moderators of CMV have been
non-top moderators. From the above timeline we see that moderator
permission types are stable, moderators always remain either top or
non-top through out their tenure. Importantly, the only current non-top
moderator is a non-human bot. All former CMV moderators still have
active accounts, half currently do not moderate any subreddits and one
third moderate at least one subreddit but are not connected to the
co-moderation network. Only three former moderators are part of the
current co-moderation network, suggesting a clearer distinction between
former and current moderators.

\textbf{Subreddit Network Statistics}

\begin{verbatim}
<tr>
    <td>{{output[sub1]['desc_table']}}</td>
    <td>{{output[sub2]['desc_table']}}</td>
</tr>
\end{verbatim}

    \begin{example}
  Figure~\ref{fig:hasse} depicts ...
\begin{figure}[twomode_net_td.png]
\centering
\begin{tikzpicture}
    \tikzstyle{every node}=[draw,circle,fill=black,minimum size=4pt,
                            inner sep=0pt]
    [bunch of lines for the picture]    
\end{tikzpicture}
\caption{Hasse diagram}
\label{fig:hasse}
\end{figure}
\end{example}

    \paragraph{Moderator Type Homophily}\label{moderator-type-homophily}

The tables above show the Krackhardt E/I Ratio for each of the one-mode
networks. This is only relevant in the moderator networks, where
moderators are grouped by their moderator type (current/former,
top/non-top). Edges are 'within type' if the nodes share the same
moderator status, example they are both former tops. Edges are 'between
types' if the nodes have different moderator types, example one current
top and one former non-top. Thus the E-I index is this application was
computed by:

\[\frac{\textrm{number of between type edges} - \textrm{number of within type edges}}{\textrm{total number of edges}}\]

An E-I index of -1 means all edges are within group (homophily), +1
means all edges are between group (heterophily). The r/The\_Donald
moderator network was an E-I index of 0.27 - there are slightly more
ties external to groups than internal, suggesting some heterophily
between moderator types. The r/changemyview moderator network was an E-I
index of -0.85 - there are many more ties interal to groups than
external, suggesting significant homophily between moderator types.
However, this is to be expected given that 1/2 of all redditors to ever
moderate r/changemyview (and 63\% of those who are still active
moderators of some kind) belong to the current top group. Current top
moderators thus have greater opportunity to form edges with other
current tops.

    \textbf{Two-Mode Networks}

\begin{verbatim}
<tr>
    <th  style="text-align: center">{{subname1}}</th>
    <th  style="text-align: center">{{subname2}}</th>
</tr>
<tr>
    <td><img src='twomode_net_td.png'></td>
    <td><img src='twomode_net_cmv.png'></td>
</tr>
\end{verbatim}

    \textbf{One-Mode Moderator Networks}

As more than half of users to ever moderate CMV are current top
moderators, it is obvious that they will form the majority of the
current co-moderation network (87\%). Compared to the TD co-moderation
network, there is much less opportunity for variety in the status of
moderators in the network. However, an examination of the subreddit
network will determine which subreddits connect the 3 former moderators
to the network, and which few subreddits seem to densely connect most
current moderators. I also seek to account for the temporal aspect for
current moderators, by seeing if those who joined at the same time show
any tendency towards other similar subreddits.

\begin{verbatim}
<tr>
    <th  style="text-align: center">{{subname1}}</th>
    <th  style="text-align: center">{{subname2}}</th>
</tr>
<tr>
    <td><img src="mod_net_td.png"></td>
    <td><img src="mod_net_cmv.png"></td>
</tr>
\end{verbatim}

    \textbf{One-Mode Subreddit Network}

\begin{verbatim}
<tr>
    <th  style="text-align: center">{{subname1}}</th>
    <th  style="text-align: center">{{subname2}}</th>
</tr>
<tr>
    <td><img src="sub_net_td.png"></td>
    <td><img src="sub_net_cmv.png"></td>
</tr>
\end{verbatim}

    \section{Former Next Steps}\label{former-next-steps}

\subsection{Timeline}\label{timeline}

I plan to then use text analysis techniques to examine whether shifts in
the moderator network correspond to shifts in the informal community
discussions and formal moderator controlled standards of behaviour --
i.e. the implicit and explicit sources of community norms on
r/The\_Donald and r/ChangeMyView. My first step is to look deeper into
moderation eras transition periods. Then looking at top posts and
comments for sentiment and text analysis, i.e. implicit changes in
subreddit norms and changes in subreddit formal documents (ex wikis,
sidebar) for explicit changes in norms

\subsection{Co-moderation network}\label{co-moderation-network}

I will extend my analysis of the co-moderation network to consider
possible content clusters as outlined in hypothesis 2B. This may take
the form of two-mode blockmodeling of moderators by subreddits. However
I would first like to better address the issue of the longitudinal
nature of the moderator data. Dichotomising moderators as current/former
loses meaning of possible `moderation eras'. At present I only have data
on the subreddits currently moderated while it is more important which
subreddits formers moderators co-moderated at the time they moderated
CMV or TD.


    % Add a bibliography block to the postdoc
    
    
    
    \end{document}
